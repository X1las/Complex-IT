\documentclass[12pt,a4paper]{article}

% Required packages
\usepackage[utf8]{inputenc}
\usepackage[T1]{fontenc}
\usepackage{geometry}
\usepackage{graphicx}
\usepackage{amsmath}
\usepackage{amsfonts}
\usepackage{amssymb}
\usepackage{hyperref}
\usepackage{listings}
\usepackage{xcolor}
\usepackage{fancyhdr}
\usepackage{titlesec}
\usepackage{tocloft}
\usepackage{enumitem}
\usepackage{booktabs}
\usepackage{float}
\usepackage{subcaption}
\usepackage{url}
\usepackage{cite}

% Page geometry
\geometry{left=3cm, right=3cm, top=3cm, bottom=3cm}

% Fix header height issue
\setlength{\headheight}{14.5pt}

% Code listing style
\definecolor{codegreen}{rgb}{0,0.6,0}
\definecolor{codegray}{rgb}{0.5,0.5,0.5}
\definecolor{codepurple}{rgb}{0.58,0,0.82}
\definecolor{backcolour}{rgb}{0.95,0.95,0.92}

\lstdefinestyle{mystyle}{
    backgroundcolor=\color{backcolour},   
    commentstyle=\color{codegreen},
    keywordstyle=\color{magenta},
    numberstyle=\tiny\color{codegray},
    stringstyle=\color{codepurple},
    basicstyle=\ttfamily\footnotesize,
    breakatwhitespace=false,         
    breaklines=true,                 
    captionpos=b,                    
    keepspaces=true,                 
    numbers=left,                    
    numbersep=5pt,                  
    showspaces=false,                
    showstringspaces=false,
    showtabs=false,                  
    tabsize=2
}

\lstset{style=mystyle}

% Header and footer
\pagestyle{fancy}
\fancyhf{}
\rhead{\thepage}
\lhead{Full-Stack Project Report}
\renewcommand{\headrulewidth}{0.4pt}

% Hyperlink setup
\hypersetup{
    colorlinks=true,
    linkcolor=blue,
    filecolor=magenta,      
    urlcolor=cyan,
    citecolor=black,
}

% Title page information
\title{
    \LARGE \textbf{Full-Stack Computer Science Project} \\
    \Large \textit{Project Report}
}
\author{
    Your Name \\
    Student ID: XXXXXXXX \\
    \texttt{your.email@university.edu}
}
\date{\today}

\begin{document}

% Title page
\maketitle
\thispagestyle{empty}

\newpage

% Abstract
\begin{abstract}
This report presents a comprehensive overview of a full-stack computer science project, detailing the design, implementation, and evaluation of a web application system. The project encompasses frontend development, backend architecture, database design, and deployment strategies. Key technologies include [list your main technologies here]. The system addresses [briefly describe the problem your project solves] and demonstrates proficiency in modern software development practices.
\end{abstract}

\newpage

% Table of Contents
\tableofcontents
\newpage

% List of Figures (optional)
\listoffigures
\newpage

% List of Tables (optional)
\listoftables
\newpage

\section{Introduction}

\subsection{Project Overview}
This section provides a high-level overview of the project, including its purpose, scope, and main objectives.

\subsection{Problem Statement}
Describe the specific problem or need that your project addresses. Include:
\begin{itemize}
    \item Background context
    \item Specific challenges to be solved
    \item Target audience
    \item Success criteria
\end{itemize}

\subsection{Project Objectives}
\begin{enumerate}
    \item Primary objective 1
    \item Primary objective 2
    \item Secondary objectives
\end{enumerate}

\subsection{Report Structure}
This report is organized as follows: Section~\ref{sec:literature} reviews related work, Section~\ref{sec:methodology} describes the development methodology, Section~\ref{sec:system-design} presents the system architecture, Section~\ref{sec:implementation} details the implementation, Section~\ref{sec:testing} covers testing strategies, Section~\ref{sec:results} presents results and evaluation, and Section~\ref{sec:conclusion} concludes with lessons learned and future work.

\section{Literature Review and Related Work}
\label{sec:literature}

\subsection{Existing Solutions}
Review existing solutions in the problem domain. Compare and contrast different approaches.

\subsection{Technology Stack Analysis}
Discuss the technologies chosen for the project:
\begin{itemize}
    \item Frontend frameworks and libraries
    \item Backend technologies and frameworks
    \item Database management systems
    \item Development tools and environments
    \item Deployment and hosting platforms
\end{itemize}

\subsection{Best Practices and Design Patterns}
Describe relevant software engineering principles and patterns applied in the project.

\section{Methodology}
\label{sec:methodology}

\subsection{Development Approach}
Describe your development methodology (Agile, Waterfall, etc.) and justify your choice.

\subsection{Project Management}
\begin{itemize}
    \item Timeline and milestones
    \item Risk assessment and mitigation strategies
    \item Version control and collaboration tools
    \item Testing strategies
\end{itemize}

\subsection{Quality Assurance}
Outline your approach to ensuring code quality, including:
\begin{itemize}
    \item Code review processes
    \item Automated testing
    \item Continuous integration/deployment
    \item Documentation standards
\end{itemize}

\section{System Design and Architecture}
\label{sec:system-design}

\subsection{System Architecture Overview}
Provide a high-level architecture diagram and explanation of the system components.

\begin{figure}[H]
    \centering
    % \includegraphics[width=0.8\textwidth]{images/system-architecture.png}
    \caption{System Architecture Diagram}
    \label{fig:system-architecture}
\end{figure}

\subsection{Frontend Design}
\subsubsection{User Interface Design}
Describe the UI/UX design principles and wireframes.

\subsubsection{Frontend Architecture}
Detail the frontend component structure, state management, and routing.

\subsection{Backend Design}
\subsubsection{API Design}
Document the REST API endpoints or GraphQL schema.

\subsubsection{Business Logic}
Explain the core business logic and service layer architecture.

\subsubsection{Authentication and Authorization}
Describe the security implementation for user authentication and access control.

\subsection{Database Design}
\subsubsection{Entity Relationship Diagram}
\begin{figure}[H]
    \centering
    % \includegraphics[width=0.9\textwidth]{images/erd-diagram.png}
    \caption{Entity Relationship Diagram}
    \label{fig:erd}
\end{figure}

\subsubsection{Database Schema}
Provide detailed table structures and relationships.

\subsubsection{Data Flow}
Explain how data flows through the system from frontend to backend to database.

\section{Implementation}
\label{sec:implementation}

\subsection{Frontend Implementation}
\subsubsection{Key Components}
Describe the main frontend components and their functionality.

\begin{lstlisting}[caption=Example React Component]
// Example React Component
import React, { useState } from 'react';

const UserProfile = ({ user }) => {
  const [isEditing, setIsEditing] = useState(false);
  
  return (
    <div className="user-profile">
      <h2>{user.name}</h2>
      <p>{user.email}</p>
      {isEditing && (
        <button onClick={() => setIsEditing(false)}>
          Save Changes
        </button>
      )}
    </div>
  );
};

export default UserProfile;
\end{lstlisting}

\subsubsection{State Management}
Explain how application state is managed (Redux, Context API, etc.).

\subsubsection{Responsive Design}
Describe how the application adapts to different screen sizes and devices.

\subsection{Backend Implementation}
\subsubsection{Server Setup}
Detail the server configuration and middleware setup.

\begin{lstlisting}[caption=Express Server Setup]
// Express Server Setup
const express = require('express');
const cors = require('cors');
const app = express();

// Middleware
app.use(cors());
app.use(express.json());
app.use(express.urlencoded({ extended: true }));

// Routes
app.get('/api/users', (req, res) => {
  res.json({ message: 'Users endpoint' });
});

const PORT = process.env.PORT || 3000;
app.listen(PORT, () => {
  console.log(`Server running on port ${PORT}`);
});
\end{lstlisting}

\subsubsection{API Endpoints}
Document key API endpoints with examples.

\subsubsection{Database Integration}
Explain the ORM/ODM usage and database connection management.

\subsection{Security Implementation}
\begin{itemize}
    \item Input validation and sanitization
    \item Authentication mechanisms
    \item Authorization and role-based access control
    \item Data encryption and protection
    \item CORS and other security headers
\end{itemize}

\subsection{Performance Optimization}
Describe optimization techniques implemented:
\begin{itemize}
    \item Frontend optimization (code splitting, lazy loading)
    \item Backend optimization (caching, query optimization)
    \item Database indexing and query optimization
\end{itemize}

\section{Testing and Quality Assurance}
\label{sec:testing}

\subsection{Testing Strategy}
Outline your comprehensive testing approach.

\subsection{Unit Testing}
\begin{itemize}
    \item Frontend component testing
    \item Backend function testing
    \item Test coverage metrics
\end{itemize}

\begin{lstlisting}[caption=Example Unit Test]
// Example Unit Test using Jest
describe('User Authentication', () => {
  test('should authenticate user with valid credentials', async () => {
    const userData = {
      email: 'test@example.com',
      password: 'password123'
    };
    
    const result = await authenticateUser(userData);
    
    expect(result.success).toBe(true);
    expect(result.token).toBeDefined();
    expect(result.user.email).toBe(userData.email);
  });
  
  test('should reject invalid credentials', async () => {
    const userData = {
      email: 'test@example.com',
      password: 'wrongpassword'
    };
    
    const result = await authenticateUser(userData);
    
    expect(result.success).toBe(false);
    expect(result.error).toBe('Invalid credentials');
  });
});
\end{lstlisting}

\subsection{Integration Testing}
\begin{itemize}
    \item API endpoint testing
    \item Database integration testing
    \item End-to-end workflow testing
\end{itemize}

\subsection{User Acceptance Testing}
Describe user testing procedures and feedback incorporation.

\subsection{Performance Testing}
\begin{itemize}
    \item Load testing results
    \item Performance benchmarks
    \item Optimization outcomes
\end{itemize}

\section{Deployment and DevOps}
\label{sec:deployment}

\subsection{Deployment Architecture}
Describe your deployment setup and infrastructure.

\subsection{Continuous Integration/Continuous Deployment}
\begin{itemize}
    \item CI/CD pipeline setup
    \item Automated testing in deployment
    \item Environment management (development, staging, production)
\end{itemize}

\subsection{Monitoring and Logging}
\begin{itemize}
    \item Application monitoring tools
    \item Error tracking and logging
    \item Performance monitoring
\end{itemize}

\section{Results and Evaluation}
\label{sec:results}

\subsection{Functional Requirements Evaluation}
Assess how well the system meets the original functional requirements.

\begin{table}[H]
\centering
\begin{tabular}{@{}lcc@{}}
\toprule
Requirement & Target & Achieved \\ \midrule
Feature 1 & 100\% & 95\% \\
Feature 2 & 100\% & 100\% \\
Feature 3 & 100\% & 90\% \\
\bottomrule
\end{tabular}
\caption{Functional Requirements Achievement}
\label{tab:functional-requirements}
\end{table}

\subsection{Non-Functional Requirements Evaluation}
\begin{itemize}
    \item Performance metrics
    \item Security assessment
    \item Usability evaluation
    \item Scalability analysis
\end{itemize}

\subsection{User Feedback}
Present results from user testing and feedback sessions.

\subsection{Performance Metrics}
\begin{table}[H]
\centering
\begin{tabular}{@{}lcc@{}}
\toprule
Metric & Target & Achieved \\ \midrule
Page Load Time & <2s & 1.5s \\
API Response Time & <500ms & 300ms \\
Concurrent Users & 100 & 150 \\
\bottomrule
\end{tabular}
\caption{Performance Metrics}
\label{tab:performance-metrics}
\end{table}

\section{Challenges and Solutions}
\label{sec:challenges}

\subsection{Technical Challenges}
Describe major technical challenges encountered and how they were resolved.

\subsection{Project Management Challenges}
Discuss any project management or timeline challenges and solutions.

\subsection{Learning Outcomes}
Reflect on what was learned during the project development process.

\section{Future Work and Improvements}
\label{sec:future-work}

\subsection{Planned Enhancements}
\begin{itemize}
    \item Additional features to be implemented
    \item Performance improvements
    \item Scalability enhancements
    \item User experience improvements
\end{itemize}

\subsection{Technical Debt}
Acknowledge any technical debt and plans for addressing it.

\subsection{Scalability Considerations}
Discuss how the system could be scaled for larger user bases or data volumes.

\section{Conclusion}
\label{sec:conclusion}

\subsection{Project Summary}
Summarize the key achievements and deliverables of the project.

\subsection{Objectives Assessment}
Evaluate how well the original objectives were met.

\subsection{Personal Reflection}
Reflect on the development experience, skills gained, and lessons learned.

\subsection{Final Thoughts}
Conclude with thoughts on the project's success and potential impact.

% References
\bibliographystyle{plain}
\bibliography{references}
% Note: Create a references.bib file for your bibliography

% Appendices
\appendix

\section{Code Repository}
\label{app:code-repo}
Link to the project repository: \url{https://github.com/username/project-name}

\section{Installation and Setup Guide}
\label{app:installation}
Provide step-by-step instructions for setting up the development environment and running the application.

\section{API Documentation}
\label{app:api-docs}
Include detailed API documentation or link to external documentation.

\section{Database Schema Details}
\label{app:db-schema}
Provide complete database schema with all table definitions.

\section{User Manual}
\label{app:user-manual}
Include screenshots and instructions for using the application.

\end{document}